\documentclass[]{article}

%opening
\title{Overview of MA4413 2013 Paper}
\author{UL Mathematics \& Statistics}

\begin{document}

\maketitle

\begin{itemize}
\item[(1a)] (6 Marks) Basic Probability
\item[(1b)] (6 Marks) Descriptive Statistics
\item[(1c)] (8 Marks) Discrete RVs
%------------------------ %
\item[(2a)] (12 Marks)Poisson and Exponential
\item[(2b)] (8 Marks) Normal Distribution
%------------------------ %
\item[(3a)] (12 Marks) Confidence interval and Theory Questions
\item[(3b)] (8 Marks) Hypothesis Testing ( with \texttt{R})
%------------------------ %
\item[(4a)] (12 Marks)Hypothesis Testing (2 Sample)
\item[(4b)] (8 Marks) Accuracy, Precision, Recall and F measure
%------------------------ %
\item[(5a)] (6 Marks) Huffman Coding
\item[(5b)] (8 Marks) Entropy Information / Calculations
\item[(5c)] (6 Marks) Binary Channels
\end{itemize}

%-----------------------------------%
\newpage
\section{Probability Questions}
%--------------------------------------------------------------%
\subsection*{Part 2A}

(a)	Suppose 40\% of employees in a large company favour unionisation.  A poll of 10 employees in this company is taken.  
 \begin{itemize}
\item[(i)] (1 Mark) 	What is the probability that 4 or more employees polled favour unionisation? 
\item[(ii)] (1 Mark) 	What is the probability that less than 2 employees polled favour unionisation?
\item[(iii)] (1 Mark) 	What is the probability that exactly 5 employees polled favour unionisation?
\item[(iv)] (1 Mark) 	What is the mean and variance for this distribution?
\end{itemize}										       

%--------------------------------------------------------------%
\subsection*{Part 2B}
(b)	A student is practising for an upcoming high jump event.  The height that she will clear each time she jumps is normally distributed with a 
mean of 72 inches and a standard deviation of 4 inches.  
\begin{itemize}
\item[(i)] (1 Mark) 	What is the probability that the jumper will clear 76 inches or higher on a single jump?
\item[(ii)] (1 Mark) 	What is the probability that the height she jumps is between 68 and 76 inches on a single jump?
\item[(iii)] (1 Mark) 	What is the minimum height she must jump in order for the jump to be in the highest 10%?
\end{itemize}
    

%--------------------------------------------------------------%
\subsection*{Part 2C}
(c)	Telephone calls coming in to a busy switchboard follow a Poisson distribution with 3 calls expected in a one minute period.  
The switchboard operator can answer at most 3 calls in a one minute period; the fourth and succeeding calls receive a busy signal.
\begin{itemize} 
\item[(i)] (1 Mark) 	   Find the probability of receiving a busy signal.
\item[(ii)] (1 Mark) 	   The switchboard operator leaves the switchboard unattended for 2 minutes.  
\item[(iii)] (1 Mark) What is the probability that exactly 1 call will be missed during that 2 minute period?
\end{itemize}       

(d)	In what circumstances can the Poisson distribution be used to approximate the Binomial distribution?
      


%--------------------------------------------------------------%
\subsection*{Part 2D}


(a)  	Four  per cent of PCB boards purchased over the Internet from the Far East have some defect.  
From a large consignment of boards, 50 are chosen at random.  
What is the probability that:
\begin{itemize}
\item[(i)] (1 Mark) 	3 or more boards have some defect?
\item[(ii)] (1 Mark) 	Exactly 2 boards are defective?
\item[(ii)] (1 Mark) 	Less than 3 boards are defective?
\item[(iv)] (1 Mark) 	If more than 5 boards from the 50 were defective what action would you take? (justify)
\end{itemize} 
(b) 	Flaws occur in a hard wood timbers  at the rate of 1.5 per linear  metre section.  Calculate the probability that:
\begin{itemize}
\item[(i)] (1 Mark) 	3 or more flaws will occur in a 3  metre length 
\item[(ii)] (1 Mark) 	Exactly 4 flaws will occur in a 10 metre length
\item[(iii)] (1 Mark) 	8 or less flaws will occur in a 6  metre length 
\end{itemize} 

(c)	There is a constant probability of 0.05  that the power supply in telecoms network will not start.  
You are requested to calculate the probability that the power supply will fail the 5th time it is activated.

\newpage
\section*{Question 5A - 20 marks}
%Question 5. (20 marks)
(a) Consider the binary channel in the figure below.

\begin{itemize}
\item[(i)] (1 Mark) Determine the channel matrix of the channel
\item[(i)] (1 Mark)  Find P(Y1) and P(Y2) when P(X1) = 0:7 and P(X2) = 0:3
\item[(i)] (1 Mark)  Find the joint probabilities P(X1; Y1) and P(X2; Y2).
\end{itemize}
%----------------------------------------------------- %
(b) A discrete memoryless source X has five symbols $\{x_1; x_2; x_3; x_4; x_5\}$ with prob-
abilities P(x1) = 0:40 , P(x2) = 0:25, P(x3) = 0:15, P(x4) = 0:12 and
P(x5) = 0:08.
\begin{verbatim}
i. (4 marks) Construct a Huffman code for X.
ii. (4 marks) Calculate the efficiency of the code.
iii. (2 marks) Calculate the redundancy of the code.
\end{verbatim}
\end{document}
