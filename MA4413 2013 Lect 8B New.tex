

\noindent \textbf{p-values}
\begin{itemize}
\item In hypothesis tests, the difference between the observed value and the parameter value specified by $H_0$ is computed and the probability of obtaining a difference this large or large is calculated.
\item The probability of obtaining data as extreme, or more extreme, than the expected value under the null hypothesis is called the \textbf{\emph{p-value}}.
\item It is not the probability of the null hypothesis itself.
\item Suppose if the probability value is $0.0175$, this does not mean that the probability that the null hypothesis is either true (or false) is $0.0175$.

\end{itemize}

%--------------------------------------------------------------------------------------%

\noindent \textbf{p-values}
\begin{itemize}
\item the p-value means that the probability of obtaining data as different or more different from the null hypothesis as those obtained in the experiment is $0.0175$.
\item If the p-value is less than the specified significance level, adjusted for the number of tails, then we reject the null hypothesis.
\[\mbox{ is p-value} \leq \frac{\alpha}{k} \mbox{?}\]
\end{itemize}



\noindent \textbf{The Hypothesis Testing Procedure }
The second procedures is very similar to the first, but is more practicable for written exams, so we will use this one more. The first two steps are the same.

\begin{itemize}
\item Formally write out the null and alternative hypotheses (already described).
\item Compute the test statistic
\item Determine the \emph{\textbf{critical value}} (described shortly)
\item Make a decision based on the critical value.
\end{itemize}




\noindent \textbf{Hypothesis Tests for single samples}

\begin{itemize}
\item We could have inference procedures for single sample studies. We would base an argument on the either the sample mean or sample proportion as appropriate.
\item A hypothesis test can be used to determine how ``confident" we can be with our data in making that statements.
\item The lower the significance level (The margin for Type I error) the stronger our data must be.
\item Large samples lead to more confident conclusion.
\end{itemize}


\noindent \textbf{Hypothesis Tests for single samples}
\begin{itemize}
\item We could have either hypothesis test for the sample mean or the sample proportion, to test a statement about the population as a whole (i.e something about the population mean)
\item We make our argument in the form of the null and alternative hypotheses. 
\item The Hypothesis testing procedure determines the strength of evidence in making our arguments. 
\end{itemize}


\noindent \textbf{Hypothesis Tests for single samples}
\begin{itemize}
\item We simply follow the four step procedure. 
\item All of the components are the same used in confidence intervals.
\item The critical value is simply a quantile from the $Z$ or $t-$distribution.
\item The standard errors are also as before. Although when performing a hypothesis test for proportions, we use the expected value under the null hypothesis, rather than point estimate. (reason beyond scope of course.)
\end{itemize}

%%%%%%%%%%%%%%%%%%%%%%%%%%%%%%%%%%%%%%%%%%%%%%%%%%%%%%%%%%%%%%%%%%%%%%%%%%%%%%%%%%%%%%%%%%%%%%%5



%%%% Type I and Type II errors here
{
\noindent \textbf{The Paired t-test}
A paired t-test is used to compare two population means where you have two samples in
which observations in one sample can be \textbf{\emph{paired}} with observations in the other sample.\\
\bigskip
Examples of where this might occur are:
\begin{itemize}
\item Before-and-after observations on the same subjects (e.g. students’ diagnostic test
results before and after a particular module or course).
\item A comparison of two different methods of measurement or two different treatments
where the measurements/treatments are applied to the \textbf{\emph{same}} subjects.
\end{itemize}
The difference between two paired measurements is known as a \textbf{\emph{case-wise}} difference.
}



%-------------------------------------------------------------------------------------------%

\noindent \textbf{The Paired t-test}
\begin{itemize}
\item We will often be required to compute the case-wise differences, the average of those differences and the standard deviation of those difference.

\item The mean difference for a set of differences between paired observations is
\[ \bar{d} = {\sum d_i \over n }\]

\item The computational formula for the standard deviation of the differences
between paired observations is
\[s_d = \sqrt{ {\sum d_i^2 - n\bar{d}^2 \over n-1}}\]
\item It is nearly always a small sample test.
\end{itemize}



%----------------------------------------------------------------------------------------------------%
{
\noindent \textbf{The Paired t-test}
\begin{itemize}
\item $\mu_d$ mean value for the population of case-wise differences.
\item The null hypothesis is that that $\mu_d = 0$
\item Given $\bar{d}$ mean value for the sample of differences, and $s_d$ standard deviation of the differences for the paired sample data, we can compute this test in the same manner as a one-sample test for the mean
\end{itemize}
}


\end{document}
