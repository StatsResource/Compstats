\documentclass[a4]{beamer}
\usepackage{amssymb}
\usepackage{graphicx}
\usepackage{subfigure}
\usepackage{newlfont}
\usepackage{amsmath,amsthm,amsfonts}
%\usepackage{beamerthemesplit}
\usepackage{pgf,pgfarrows,pgfnodes,pgfautomata,pgfheaps,pgfshade}
\usepackage{mathptmx}  % Font Family
\usepackage{helvet}   % Font Family
\usepackage{color}

\mode<presentation> {
 \usetheme{Default} % was Frankfurt
 \useinnertheme{rounded}
 \useoutertheme{infolines}
 \usefonttheme{serif}
 %\usecolortheme{wolverine}
% \usecolortheme{rose}
\usefonttheme{structurebold}
}

\setbeamercovered{dynamic}

\title[MA4413t]{Statistics for Computing \\ {\normalsize Revision Class 13A}}
\author[Kevin O'Brien]{Kevin O'Brien \\ {\scriptsize kevin.obrien@ul.ie}}
\date{Summer 2011}
\institute[Maths \& Stats]{Dept. of Mathematics \& Statistics, \\ University \textit{of} Limerick}

\renewcommand{\arraystretch}{1.5}


\begin{document}
%----------------------------------------------------------------------------------------------------------%


\begin{frame}
\titlepage
\end{frame}

\frame{
\frametitle{Revision for Inference Procedures}
\large
\begin{itemize}
\item Definitions
\item Computing Confidence Intervals 
\item Performing Hypothesis Testing
\begin{itemize}
\item by comparing test statistics to critical values
\item by considering the p-value
\item by using the confidence interval.
\end{itemize}
\end{itemize}
}

\frame{
\frametitle{Inference : Definitions (1) }
\large
\begin{itemize}
\item Samples
\begin{itemize}
\item Sample and Population
\item Sampling error
\end{itemize}
\item Sampling Distributions
\begin{itemize}
\item Central Limit Theorem
\item Standard Error
\end{itemize}
\end{itemize}
}

\frame{
\frametitle{Inference : Definitions (2) }
\begin{itemize}

\item Underlying theory of hypothesis testing
\large
\begin{itemize}
\item The p-value
\end{itemize}

\item Hypothesis tests
\begin{itemize}
\item Null hypothesis
\item Alternative hypothesis
\end{itemize}
\item Decisions
\begin{itemize}
\item Test Statistics
\item Acceptance Region
\item Critical Regions (Rejection Region)
\end{itemize}
\end{itemize}
}


\frame{
\frametitle{Inference : Definitions (3) }
\large
\begin{itemize}

\item Types of Error
\begin{itemize}
\item Type I error (Significance)
\item Type II error (Power)
\end{itemize}


\item Important Skills
\begin{itemize}
\item Using Murdoch-Barnes table 7 to compute Quantiles / Critical Values.
\item Using Murdoch-Barnes table 3 to compute p-Values.
\end{itemize}
\end{itemize}
}

\frame{
\frametitle{Inference : Structure of a Hypothesis Test (1) }
\large
\begin{itemize}
\item Formally write out the null and Alternative Hypothesis.
\begin{itemize}
\item Denote the null as $H_0$ and the alternative as $H_1$.
\item Use the parameter values (i.e. $\mu$ and $\pi$), not the sample estimates.
\item Remember to provide a brief description of each hypothesis.
\end{itemize}
\end{itemize}
}


\frame{
\frametitle{Inference : Structure of a Hypothesis Test (2) }
\large

\begin{itemize}
\item Compute the Test Statistic ($TS$)
\begin{itemize}
\item You will need to compute the value for Standard Error (See back of exam paper).
\item The general structure is 
\[ {\mbox{observed value} - \mbox{null value} \over \mbox{Standard Error}}  \]
\item The p-value is computed as $P(Z \geq |TS|)$ (from Murdoch Barnes 3). \\N.B. p-value is for large samples only.
\end{itemize}
\end{itemize}
}



\frame{
\frametitle{Inference : Structure of a Hypothesis Test (3) }
\large

\begin{itemize}
\item Determine the Critical Value
\begin{itemize}
\item You will need to know the sample size ($n$), the significance ($\alpha$) , and the number of tails ($k$).
\item In this module, $\alpha$ = 0.05 and $k=2$ always.
\item Depending on the sample size the degrees of freedom is $\nu = n-1$ 9 when $n \leq 30$ or $\nu = \infty$ when $n > 30$
\end{itemize}
\end{itemize}
}


\frame{
\frametitle{Inference : Structure of a Hypothesis Test (4) }

\begin{itemize}
\item Making a decision (Critical Value) :  Is the absolute value of the Test Statistic greater than the Critical Value?
\begin{itemize}
\item If $|TS| > CV $ We reject the null hypothesis.
\item If $|TS| \leq CV $ We fail to reject the null hypothesis.
\end{itemize}
\end{itemize}
}

\frame{
\frametitle{Inference : Structure of a Hypothesis Test (4) }
\begin{itemize}
\item Making a decision (p-value) : \\ Is the p-value less than than the critical threshold $\alpha / k$.?
\begin{itemize}
\item If p-value $< \alpha /k $ :  We reject the null hypothesis.
\item If p-value $\geq \alpha /k $ : We fail to reject the null hypothesis.
\end{itemize}
\end{itemize}



}

\frame{
\frametitle{Inference : Confidence Intervals }

Basic Structure

\[ \mbox{Observed value} \pm [\mbox{Quantile}\times \mbox{Standard Error}] \] 



}



\frame{
\frametitle{Inference : Paired values}

\begin{itemize}

\item Know how to compute case-wise differences.
\item Know how to compute the mean of the case-wise differences (see formulae). 
\item Know how to compute the standard deviation of the casewise differences (see formulae).
\end{itemize} 
}
\end{document}