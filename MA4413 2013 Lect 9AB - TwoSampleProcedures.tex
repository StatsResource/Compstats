
\textbf{Two Sample Procedures}
\begin{itemize}
\item So far we have looked at single sample procedures.
\item We can generalise our methodologies for comparing two samples.
\item Our point estimates are typically differences in sample statistics. i.e.
\[ \bar{X}_1 - \bar{X}_2\]
\[\hat{p}_1 - \hat{p}_2 \]
\item We can use these point estimates to make inference on differences for populations.
\[ \mu_1 - \mu_2\]
\[ \pi_1 - \pi_2 \]
\end{itemize}


\noindent \textbf{Two Sample Procedures - Confidence Intervals}
\begin{itemize}
\item When computing confidence intervals, all that is required is the calculation of the appropriate standard error value.
\item The two-sample standard error calculations contain statistical information from both samples.
\item See the formula sheet. (Practice with these calculaions will take place in next week's tutorials.)
\item There are some minor issues that will arise with each type of procedure. These will be explained in relevant examples.
\end{itemize}


\noindent \textbf{Two Sample Procedures - small samples and degrees of freedom}
\begin{itemize}
\item Let $n_1$ and $n_2$ be the sample sizes of two samples.
\item When deciding whether to use the large sample approach or the small sample appproach, we will use the following rule of thumb: \\
Small sample : $n_1+n_2 \leq 30$\\
Large sample : otherwise\\
\item For small samples the appropriate degrees of freedom is $(n_1-1) + (n_2-1)$ i.e. $n_1 + n_2-2$ 
\end{itemize}


\textbf{Two Sample Procedures - small samples and degrees of freedom}
\begin{itemize}
\item When performing hypothesis tests, we are usually interested in determining whether or not the population parameters can be considered equal for both populations.
\item Another way of expressing this is that the difference in population parameters is 0. 
\item The following two hypotheses are directly equivalent.
\[ H_o :  \mu_1 = \mu_2\]
\[ H_o :  \mu_1 - \mu_2 =0\]
\item Equivalently, for proportions:
\[ H_o :  \pi_1 = \pi_2\]
\[ H_o :  \pi_1 - \pi_2 =0\]
\end{itemize}


\noindent \textbf{Two Sample Procedures - Standard Errors}
\begin{itemize}
\item \textbf{Small Samples}: When computing the standard error for difference in sample means - take care to use the appropriate standard error formula. (i.e using the pooled variance calculation.)
\item \textbf{Hypothesis Tests for Propostions:} Use the aggregate proportion formula $\bar{p}$.
\item See formula sheet.
\end{itemize}



