\subsection*{Test 1. Single Sample Test for Proportions}
\begin{itemize}
\item Sample size ($n$) = 500
\item Number of successes ($x$) = 280
\item Expected value under null hypothesis (Usually $\pi$, but here as $p$)
\end{itemize}
\begin{framed}
\begin{verbatim}
> prop.test(x=280,n=500,p=0.60)

   1-sample proportions test with continuity correction

data:  280 out of 500, null probability 0.6
X-squared = 3.1688, df = 1, p-value = 0.07506
alternative hypothesis: true p is not equal to 0.6
95 percent confidence interval:
 0.5151941 0.6038700
sample estimates:
   p 
0.56 
> 
\end{verbatim}
\end{framed}
\newpage
\subsection*{Test 2. F Test for equality of variance}
\begin{itemize}
\item In this procedure, we determine whether or not two \textit{\textbf{populations}} have the same variance.
\item The assumption of equal variance of two populations underpins several inference procedures. This assumption is tested by comparing the variance of samples taken from both populations.
\item The null and alternative hypotheses are as follows:
\[ H_0: \sigma^2_1 = \sigma^2_2 \]
\[ H_1: \sigma^2_1 \neq \sigma^2_2 \]
\end{itemize}
\begin{framed}
\begin{verbatim}
> var.test(X,Y)

        F test to compare two variances

data:  X and Y
F = 2.5122, num df = 9, denom df = 9, p-value = 0.1862
alternative hypothesis: true ratio of variances
  is not equal to 1
95 percent confidence interval:
  0.6239986 10.1141624
sample estimates:
ratio of variances 
          2.512215 
\end{verbatim}
\end{framed}

\subsection*{Test 3. Shapiro Wilk's Test for Normality}

\begin{framed}
For each procedures, identify the null and alternative hypotheses, the $p-$value, and your conclusion for this test.

\noindent For one-tailed tests:
\begin{itemize}
\item If the $p-$value is less than 0.05 : reject the null hypothesis.
\item If the $p-$value is greater than 0.05 : fail to reject the null hypothesis.
\end{itemize}
\end{framed}

\begin{itemize}
\item We will often be required to determine whether or not a data set is normally distributed.
This assumption underpins many statistical models.
\item The null hypothesis is that the data set is normally distributed.
\item The alternative hypothesis is that the data set is not normally distributed.
\item One procedure for testing these hypotheses is the Shapiro-Wilk test, implemented in \texttt{R} using the command \texttt{shapiro.test()}.
\end{itemize}
\begin{framed}
\begin{verbatim}
> shapiro.test(X)

        Shapiro-Wilk normality test

data:  X
W = 0.9849, p-value = 0.1012


\end{verbatim}
\end{framed}
\subsection*{Test 4. Grubbs' Test for Determining an Outlier}

\begin{framed}
For each procedures, identify the null and alternative hypotheses, the $p-$value, and your conclusion for this test.

\noindent For two-tailed tests:
\begin{itemize}
\item If the $p-$value is less than 0.025 : reject the null hypothesis.
\item If the $p-$value is greater than 0.025 : fail to reject the null hypothesis.
\end{itemize}
\end{framed}
The Grubbs' test is used to determine if there are any outliers in a data set.\\ \bigskip

\noindent There is no agreed formal definition for an outlier. The definition of outlier used for this procedure is a value that unusually distant from the rest of the values (For the sake of clarity , we shall call this type of outlier a \textbf{Grubbs' Outlier}). Consider the following data set: is the lowest value 4.01 an outlier?
\begin{center}
\begin{verbatim}
6.98 8.49 7.97 6.64
8.80 8.48 5.94 6.94
6.89 7.47 7.32 4.01
\end{verbatim}
\end{center}


\begin{framed}
\begin{verbatim}
> grubbs.test(x, two.sided=T)
Grubbs test for one outlier
data: x
G = 2.4093, U = 0.4243, p-value = 0.05069
alternative hypothesis: lowest value 4.01 is an outlier
\end{verbatim}
\end{framed}
%---------------------------------------%