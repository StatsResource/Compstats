 \documentclass[a4paper,12pt]{article}
%%%%%%%%%%%%%%%%%%%%%%%%%%%%%%%%%%%%%%%%%%%%%%%%%%%%%%%%%%%%%%%%%%%%%%%%%%%%%%%%%%%%%%%%%%%%%%%%%%%%%%%%%%%%%%%%%%%%%%%%%%%%%%%%%%%%%%%%%%%%%%%%%%%%%%%%%%%%%%%%%%%%%%%%%%%%%%%%%%%%%%%%%%%%%%%%%%%%%%%%%%%%%%%%%%%%%%%%%%%%%%%%%%%%%%%%%%%%%%%%%%%%%%%%%%%%
\usepackage{eurosym}
\usepackage{vmargin}
\usepackage{amsmath}
\usepackage{multicol}
\usepackage{graphics}
\usepackage{epsfig}
\usepackage{framed}
\usepackage{subfigure}
\usepackage{fancyhdr}

\setcounter{MaxMatrixCols}{10}
%TCIDATA{OutputFilter=LATEX.DLL}
%TCIDATA{Version=5.00.0.2570}
%TCIDATA{<META NAME="SaveForMode" CONTENT="1">}
%TCIDATA{LastRevised=Wednesday, February 23, 2011 13:24:34}
%TCIDATA{<META NAME="GraphicsSave" CONTENT="32">}
%TCIDATA{Language=American English}

%\pagestyle{fancy}
\setmarginsrb{20mm}{0mm}{20mm}{25mm}{12mm}{11mm}{0mm}{11mm}
%\lhead{MA4413 2013} \rhead{Mr. Kevin O'Brien}
%\chead{Midterm Assessment 1 }
%\input{tcilatex}

\begin{document}

\begin{center}
       \includegraphics[scale=0.60]{shieldtransparent2}
\end{center}

\begin{center}
\vspace{1cm}
\large \bf {FACULTY OF SCIENCE AND ENGINEERING} \\[0.5cm]
\normalsize DEPARTMENT OF MATHEMATICS AND STATISTICS \\[1.25cm]
\large \bf {END OF SEMESTER EXAMINATION} \\[1.5cm]
\end{center}

\begin{tabular}{ll}
MODULE CODE: MA4413 & SEMESTER: Autumn 2013\\[1cm]
MODULE TITLE: Statistics for Computing & DURATION OF EXAM: 2.5 hours \\[1cm]
LECTURER: Kevin O'Brien & GRADING SCHEME: 100 marks\\
 & \phantom{GRADING SCHEME:} \footnotesize {70\% of total module marks}   \\[0.8cm]
EXTERNAL EXAMINER: Prof. Brendan Murphy & \\[1cm]
\\[1cm]
\end{tabular}
\begin{center}
{\bf INSTRUCTIONS TO CANDIDATES}
\end{center}

{\noindent \\ This paper is comprised of five questions, each worth 25 marks. Attempt any four questions.
\\ Scientific calculators approved by the University of Limerick can be used. 
\\ Formula sheet and statistical tables are provided.
}
\normalsize
\newpage






%\subsection*{Q6. Poisson Distribution (2 Marks) }  % 14 Marks
%Suppose that a telephone help-line receives 4 calls per hour during offices hours.
%\begin{itemize}
%\item[a.] (1 Mark) Compute the value of $m$ for a 30 minute period during office hours.
%\item[b.] (1 Mark) Compute the probability of the help-line getting exactly one call in a 30 minute period during office hours.
%\end{itemize}
%\bigskip
%\subsection*{Q7. Exponential Distribution (1 Mark)} % 15 Marks


%\begin{itemize}

%\item[a.] (1 Mark) Compute the value of $P(X \leq 482)$

%\end{itemize}

\section*{Formulae}
\subsection*{Probability}
\begin{itemize}

\item Conditional probability:
\begin{equation*}
P(B|A)=\frac{P\left( A\text{ and }B\right) }{P\left( A\right) }.
\end{equation*}


\item Bayes' Theorem:
\begin{equation*}
P(B|A)=\frac{P\left(A|B\right) \times P(B) }{P\left( A\right) }.
\end{equation*}


\item Binomial probability distribution:
\begin{equation*}
P(X = k) = ^{n}C_{k} \times p^{k} \times \left( 1-p\right) ^{n-k}\qquad \left( \text{where}\qquad
^{n}C_{k} =\frac{n!}{k!\left(n-k\right) !}. \right)
\end{equation*}

\item Poisson probability distribution:
\begin{equation*}
P(X = k) =\frac{m^{k}\mathrm{e}^{-m}}{k!}.
\end{equation*}
\end{itemize}

\subsection*{Information Theory}

\begin{itemize}
\item $I(p) = - log_{2}(p) = log_{2}(1/p)$\\

\item $I(pq) = I(p) + I(q)$\\

\item $H = - \sum_{i=1}^{m}p_{i}\; log_{2}(p_{i})$\\

\item $E(L) = \sum_{i=1}^{m} l_{i} p_{i}$\\

\item $\mbox{Efficiency} = H / E(L)$\\

\item $I(X;Y) = H(X) - H(X|Y)$\\

\item $P(C[r]) = \sum_{j=1}^{m}P(C[r]|Y=d_{j} )P(Y=d_{j} )$

\item $R = rH(X) \mbox{      (b/second)}$
\end{itemize}

\newpage
\subsection*{Confidence Intervals}
{\bf One sample}
\begin{eqnarray*} S.E.(\bar{X})&=&\frac{\sigma}{\sqrt{n}}.\\\\
S.E.(\hat{P})&=&\sqrt{\frac{\hat{p}\times(100-\hat{p})}{n}}.\\
\end{eqnarray*}
{\bf Two samples}
\begin{eqnarray*}
S.E.(\bar{X}_1-\bar{X}_2)&=&\sqrt{\frac{\sigma^2_1}{n_1}+\frac{\sigma_2^2}{n_2}}.\\\\
S.E.(\hat{P_1}-\hat{P_2})&=&\sqrt{\frac{\hat{p}_1\times(100-\hat{p}_1)}{n_1}+\frac{\hat{p}_2\times(100-\hat{p}_2)}{n_2}}.\\\\
\end{eqnarray*}
\subsection*{Hypothesis tests}
{\bf One sample}
\begin{eqnarray*}
S.E.(\bar{X})&=&\frac{\sigma}{\sqrt{n}}.\\\\
S.E.(\pi)&=&\sqrt{\frac{\pi\times(100-\pi)}{n}}
\end{eqnarray*}
{\bf Two large independent samples}
\begin{eqnarray*}
S.E.(\bar{X}_1-\bar{X}_2)&=&\sqrt{\frac{\sigma^2_1}{n_1}+\frac{\sigma_2^2}{n_2}}.\\\\
S.E.(\hat{P_1}-\hat{P_2})&=&\sqrt{\left(\bar{p}\times(100-\bar{p})\right)\left(\frac{1}{n_1}+\frac{1}{n_2}\right)}.\\
\end{eqnarray*}
{\bf Two small independent samples}
\begin{eqnarray*}
S.E.(\bar{X}_1-\bar{X}_2)&=&\sqrt{s_p^2\left(\frac{1}{n_1}+\frac{1}{n_2}\right)}.\\\\
s_p^2&=&\frac{s_1^2(n_1-1)+s_2^2(n_2-1)}{n_1+n_2-2}.\\
\end{eqnarray*}
{\bf Paired sample}
\begin{eqnarray*}
S.E.(\bar{d})&=&\frac{s_d}{\sqrt{n}}.\\\\
\end{eqnarray*}
{\bf Standard deviation of case-wise differences}
\begin{eqnarray*}
s_d = \sqrt{ {\sum d_i^2 - n\bar{d}^2 \over n-1}}.\\\\
\end{eqnarray*}
{\bf Binary Classification}
\begin{itemize}
\item $F = 2 \times \frac{\mbox{precision} \times \mbox{recall}}{ \mbox{precision} + \mbox{recall}}$

\end{itemize}
\end{document}



\textbf{Description of the Experiment}
\begin{itemize}
\item Suppose we have two bags, each containing 2 marbles. 
\item One bag has 2 red marbles and
the other has a red marble and a yellow marble. 
You pick a bag at random and then pick
one of the marbles in that bag at random. 
\item When you look at the marble, it is red. 
You
now pick the second marble from that same bag.
\end{itemize} 
\textbf{Questions}
\begin{itemize}
\item[(a)](1 Mark)  Write down the sample space for this experiment, where the outcomes are the ordered pairs of drawn marbles.
\item[(b)](1 Mark) What is the probability that this marble
is also red? Select one of the options below, with a justification for your answer.
You may justify your answer by references to sample points.
\end{itemize}
\textbf{Options for part b}
\begin{center}
\begin{multicols}{4}
\begin{itemize}
\item[(i)] 1/4
\item[(ii)] 1/3
\item[(iii)] 2/3
\item[(iv)] 1/2
\end{itemize}
\end{multicols}
\end{center}
